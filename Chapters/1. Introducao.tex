\section{Introdução} \label{section: Introducao}
\setstretch{1.5}
Foi-nos proposto trabalhar em grupo para idealizar uma aplicação e realizar a modelação de uma base de dados que a suporte.
\par \vspace{6pt}
O primeiro passo consistiu em definir o objetivo e as principais funcionalidades da aplicação. Esta etapa é crucial para o desenvolvimento, pois estabelece a base para a modelação da base de dados.
\par \vspace{6pt}
Após a definição da aplicação, o desafio seguinte foi modelar a base de dados de modo a suportar todas as funcionalidades identificadas. Utilizámos a ferramenta MySQL Workbench para desenhar o esquema da base de dados. Este processo envolveu a criação de tabelas, a definição de chaves primárias e estrangeiras, atributos e relacionamentos.
\par \vspace{6pt}
Com as funcionalidades da aplicação claramente definidas e o modelo de dados estabelecido, passámos a criar as queries necessárias para consulta, inserção, atualização e remoção de dados.
\par \vspace{6pt}
Finalmente, com a base de dados modelada e as queries prontas, criámos uma nova base de dados seguindo o modelo desenvolvido e introduzimos dados simulados. Isto permitiu-nos testar as queries e assegurar que o nosso modelo de dados funcionava corretamente.

\vspace{10pt}
% \begin{figure}[H]
%   \centering
%   % width=\textwidth para imagem da largura do texto
%   \includegraphics[scale=0.30]{Figures/0. General/open_vs_closed.png}
%   \caption{\textit{open source} contra \textit{closed source}}
%   \label{Open source vs. closed source}
% \end{figure}