\section{Consultas á base de dados} \label{section: Consultas}

Alguns exemplos de consultas que poderão ser feitas na aplicação:

\begin{itemize}
  \item \textbf{Registo de Utilizador}\\ 
  O \texttt{user\_id} é chave primária e será preenchido automaticamente de uma maneira sequencial. O  \texttt{vendor\_id} será inicializado sempre automaticamente como \texttt{NULL} pois quando um utilizador se regista inicialmente é apenas como um utilizador normal e não um vendedor. \texttt{deleted} também é inicializado automaticamente a \texttt{0}.
  \par
  Não será necessário armazenar qualquer palavra-passe ou token de autenticação, pois a estratégia de autenticação dos utilizadores será realizada através do \textit{Auth0}, uma plataforma de gestão de autenticação.
  
  \vspace{10pt}
    \begin{lstlisting}
  -- Exemplo de registo de um utilizador
  INSERT INTO user (first_name, last_name, email)
  VALUES ('José', 'Manuel', 'jose.manuel@atec.pt');
      \end{lstlisting}
  \item \textbf{Apagar utilizador}\\ 
      Apagar o utilizador com um certo \texttt{user\_id}. Em vez de utilizarmos \texttt{DELETE FROM}, o campo \texttt{deleted} será definido como \texttt{1}, o que é uma prática mais segura que mantém a integridade dos dados e preserva o histórico.
      \vspace{10pt}
      \begin{lstlisting}
  -- Apagar o utilizador com id igual a 1
  UPDATE user
  SET deleted = 1
  WHERE user_id = 1;        
        \end{lstlisting}    
  \item \textbf{Consultar email de todos os vendedores}\\ 
    Listar o \texttt{email} de todos os utilizadores que são vendedores, ou seja, que tenham um \texttt{vendor\_id} definido e que tenham o campo \texttt{deleted} definido como \texttt{0}, ou seja, que não tenham sido apagados.
    \vspace{10pt}
    \begin{lstlisting}
  -- Selecionar email de todos os utilizadores que são vendedores
  SELECT email FROM user
  WHERE deleted = 0 AND vendor_id IS NOT NULL;
      \end{lstlisting}

  \item \textbf{Consultar produtos e quais os seus vendedores}\\ 
    Listar o \texttt{product\_name} de todos os produtos e o seu \texttt{store\_name} correspondente.
    \vspace{10pt}
    \begin{lstlisting}
  -- Listar todos os produtos e qual a loja que os vende
  SELECT p.product_name, s.store_name
  FROM product p
  INNER JOIN store s ON s.store_id = p.store_id;
      \end{lstlisting}

  \item \textbf{Listar encomendas pertencentes a um utilizador}\\ 
    Listar o número da encomenda \texttt{order\_id}, o número da loja \texttt{store\_id}, o nome da loja \texttt{store\_name}, a data em que foi feita \texttt{created} e o seu estado correspondente \texttt{status}. Ordenar essas encomendas por ordem decrescente da mais recente á mais antiga.
    \vspace{10pt}
    \begin{lstlisting}
  -- Listar todas as encomendas do utilizador com id igual a 7
  SELECT o.order_id, s.store_id, s.store_name, o.created, o.status
  FROM `order` o
  INNER JOIN store s ON o.store_id = s.store_id
  WHERE o.user_id = 7
  ORDER BY created DESC;
      \end{lstlisting}

  \item \textbf{Listar encomendas pertencentes a um vendedor}\\ 
    Listar o número da encomenda \texttt{order\_id}, o nome do comprador \texttt{user.first\_name} e \texttt{user.last\_name}, e a data em que a encomenda foi feita \texttt{created} para todas as encomendas que um vendedor ainda tem para enviar, ou seja, aquelas que têm o \texttt{status} definido como "Em processamento", exibindo-os em ordem do mais antigo ao mais recente.
      \vspace{10pt}
      \begin{lstlisting}
  -- Listar todas as encomendas do vendedor com id igual a 1
  SELECT o.order_id, u.first_name, u.last_name, o.created
  FROM `order` o
  INNER JOIN user u ON o.user_id = u.user_id
  INNER JOIN store s ON o.store_id = s.store_id
  WHERE s.vendor_id = 1 AND o.status = 'Em processamento'
  ORDER BY o.created ASC;
        \end{lstlisting}

\end{itemize}






% Consulta de orders  que o vendedor fez e a quem 
% EXEMPLO de RIGHT JOIN;
% SELECTuser.user_id, user.first_name, user.last_name, orders.order_id
% FROM user
% RIGHT JOIN orders USING (user_id)
% WHERES  vendor_id =1;