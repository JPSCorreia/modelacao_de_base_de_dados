\section{Consultas á base de dados} \label{section: Consultas}

Alguns exemplos de consultas que poderão ser feitas na aplicação:

\begin{itemize}
  \item \textbf{Registo de Utilizador}\\ 
  O \texttt{user\_id} é chave primária e será preenchido automaticamente de uma maneira sequencial. O  \texttt{vendor\_id} será inicializado sempre automaticamente como \texttt{NULL} pois quando um utilizador se regista inicialmente é apenas como um utilizador normal e não um vendedor. \texttt{deleted} também é inicializado automaticamente a \texttt{0}.
  \par
  Não será necessário armazenar qualquer palavra-passe ou token de autenticação, pois a estratégia de autenticação dos utilizadores será realizada através do \textit{Auth0}, uma plataforma de gestão de autenticação.
  
  \vspace{10pt}
    \begin{lstlisting}
  -- Exemplo de registo de um utilizador
  INSERT INTO user (first_name, last_name, email)
  VALUES ('José', 'Manuel', 'jose.manuel@atec.pt');
      \end{lstlisting}
  \item \textbf{Apagar utilizador}\\ 
      Apagar o utilizador com um certo \texttt{user\_id}. Em vez de utilizarmos \texttt{DELETE FROM}, o campo \texttt{deleted} será definido como \texttt{1}, o que é uma prática mais segura que mantém a integridade dos dados e preserva o histórico.
      \vspace{10pt}
      \begin{lstlisting}
  -- Apagar o utilizador com id igual a 1
  UPDATE user
  SET deleted = 1
  WHERE user_id = 1;        
        \end{lstlisting}    
  \item \textbf{Consultar email de todos os vendedores}\\ 
    Mostrar o \texttt{email} de todos os utilizadores que são vendedores, ou seja, que tenham um \texttt{vendor\_id} definido e que tenham o campo \texttt{deleted} definido como \texttt{0}, ou seja, que não tenham sido apagados.
    \vspace{10pt}
    \begin{lstlisting}
  -- Selecionar email de todos os utilizadores que são vendedores
  SELECT email FROM user
  WHERE deleted = 0 AND vendor_id IS NOT NULL;
      \end{lstlisting}
\end{itemize}


\newpage

TODO: fazer mais esta pagina com consultas